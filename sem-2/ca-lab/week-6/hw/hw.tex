\documentclass{article}

\usepackage{listings}
\usepackage{enumitem}
\usepackage{amsmath}
\usepackage{svg}
\usepackage{hyperref}
\hypersetup{
    colorlinks=true,
    linkcolor=blue,
    filecolor=magenta,      
    urlcolor=cyan,
    pdftitle={Overleaf Example},
    pdfpagemode=FullScreen,
    }

\title{CA Lab: Homework 6}
\author{student: Dimitri Tabatadze}

\newcommand{\points}[1]{{\footnotesize{\color{red}\textit{#1 points}}}}

\definecolor{codegreen}{rgb}{0,0.6,0}
\definecolor{codegray}{rgb}{0.5,0.5,0.5}
\definecolor{codepurple}{rgb}{0.58,0,0.82}
\definecolor{backcolour}{rgb}{0.98,0.96,0.94}

\lstdefinestyle{mystyle}{
    backgroundcolor=\color{backcolour},   
    commentstyle=\color{codegreen},
    keywordstyle=\color{magenta},
    numberstyle=\tiny\color{codegray},
    stringstyle=\color{codepurple},
    basicstyle=\ttfamily\footnotesize,
    breakatwhitespace=false,         
    breaklines=true,                 
    captionpos=b,                    
    keepspaces=true,                 
    numbers=left,                    
    numbersep=5pt,                  
    showspaces=false,                
    showstringspaces=false,
    showtabs=false,                  
    tabsize=2
}

\lstset{style=mystyle}

\begin{document}
    \maketitle

    \section*{Task Description} 
    
    Write the code for ROM which has 3-bit inputs and 8-bit outputs.

    \begin{enumerate}
        \item Use Case block for the address with the following conditions: for each input, the output will be from 1 to 64. \points{50}
        \item Write the testbench and make the analyze of timing diagram. \points{40}
    \end{enumerate}

    Please, solve the problems (write the comments) \points{5}, take the clear screenshots and combine all your solutions in one pdf file, then upload in teams. \points{5}

    \section*{Solution}

    \begin{enumerate}
        \item {
            The Verilog code:
            
            \begin{lstlisting}[language=verilog]
module rom (
    input [2:0] address, 
    input sel,
    output reg [7:0] data
);

always @ (sel or address) begin
    if (~sel)
        data = 8'd00;
    else begin
        case (address)
            0: data = 8'd01;
            1: data = 8'd02;
            2: data = 8'd03;
            3: data = 8'd04;
            4: data = 8'd60;
            5: data = 8'd42;
            6: data = 8'd37;
            7: data = 8'd22;
        endcase
    end
end

endmodule\end{lstlisting}
        }
        \item{
            The code for the testbench:
            \begin{lstlisting}[language=verilog]
`include "rom.v"

module rom_tb();

reg [2:0] addr;
reg sel;
wire [7:0] data;

rom UUT(.address(addr), .sel(sel), .data(data));

initial begin
    $dumpfile("rom_tb.vcd");
    $dumpvars(0, rom_tb);
    addr = 0;
    sel = 0;
    #5; sel = 1;
    #5; addr = 1;
    #5; addr = 2;
    #5; addr = 3;
    #5; addr = 4;
    #5; sel = 0;
    #5; sel = 1;
    #5; addr = 5;
    #5; addr = 6;
    #5; addr = 7;
    #5;
    $finish;
end

endmodule\end{lstlisting}

            and the resulting timing diagram:

            \begin{figure}[h]
                \includegraphics[width=12cm]{output.png}
            \end{figure}
        }
    \end{enumerate}

    \section*{Conclusion}
    
    The task was \textit{almost} very clear. I think I did what I was supposed to do with the memory stored in the rom. I just put random integers in there.

    \section*{Reference}
    
    \begin{itemize}
        \item Me
        \item Myself
        \item I
        \item The slides (specifically \verb|CE_6.pdf|)
    \end{itemize}

\end{document}