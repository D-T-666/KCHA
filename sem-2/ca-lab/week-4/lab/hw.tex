\documentclass{article}

\usepackage{listings}
\usepackage{enumitem}
\usepackage{amsmath}
\usepackage{svg}
\usepackage{hyperref}
\hypersetup{
    colorlinks=true,
    linkcolor=blue,
    filecolor=magenta,      
    urlcolor=cyan,
    pdftitle={Overleaf Example},
    pdfpagemode=FullScreen,
    }

\title{CA Lab: Lab 4}
\author{student: Dimitri Tabatadze}

\begin{document}
    \maketitle

    \section*{Task Description} 
    
    \begin{enumerate}[label={\alph*)}]
        \item {Write the equations for the outputs. (15 points)}
        \item {Create truth table for the inputs X, Y, Z. Fill out the columns of the outputs. (10 points)}
        \item {Please, write what does this logic circuit represents? (15 points)}
        \item {Write Verilog HDL code of this logic diagram. (50 points)}
    \end{enumerate}

    \section*{Solution}
    
    \begin{enumerate}[label={\alph*)}]
        \item {
            \begin{displaymath}
                D = Z \oplus X \oplus Y
            \end{displaymath}
            \begin{displaymath}
                B = (\overline{(X \oplus Y)} \land Z) \lor (\overline{X} \land Y)
            \end{displaymath}
        }
        \item \begin{displaymath}
            \begin{array}{|c c c | c c |}
                Z & X & Y & D & B \\
                \hline
                0 & 0 & 0 & 0 & 0 \\
                0 & 0 & 1 & 1 & 1 \\
                0 & 1 & 0 & 1 & 0 \\
                0 & 1 & 1 & 0 & 0 \\
                1 & 0 & 0 & 1 & 1 \\
                1 & 0 & 1 & 0 & 1 \\
                1 & 1 & 0 & 0 & 0 \\
                1 & 1 & 1 & 1 & 1 \\
            \end{array}
        \end{displaymath}
        \item This is a full subtractor
        \item {
            \begin{lstlisting}
module mux (
    input Z,
    input X,
    input Y,
    output D,
    output B);

assign D = Z ^ X ^ Y;
assign B = (~(X ^ Y) & Z) | (~X & Y);

endmodule
            \end{lstlisting}
        }
    \end{enumerate}

    \section*{Conclusion}
    the homework has concluded here.
    \section*{Reference}
    there is nothing to reference.

\end{document}