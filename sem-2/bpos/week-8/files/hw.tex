\documentclass{article}

\usepackage[margin=3cm,top=3cm]{geometry}
\usepackage{listings}
\usepackage{enumitem}
\usepackage{amsmath}
\usepackage{float}
\usepackage{svg}
\usepackage{fancyvrb}
\usepackage{hyperref}
\usepackage{refcount}
\usepackage{mips}
\definecolor{codegreen}{rgb}{0,0.6,0}
\definecolor{codegray}{rgb}{0.5,0.5,0.5}
\definecolor{keyword}{rgb}{0.1,0.1,0.4}
\definecolor{codepurple}{rgb}{0.58,0,0.82}
\definecolor{red}{rgb}{0,0.7,0.3}
\lstdefinestyle{mystyle}{
    commentstyle=\color{codegreen},
    keywordstyle=\color{keyword},
    numberstyle=\tiny\color{codegray},
    stringstyle=\color{codepurple},
    basicstyle=\ttfamily\footnotesize,
    captionpos=b,
    keepspaces=true,
    numbers=left,
    showspaces=false,
    showstringspaces=false,
    frame=single,
    numbersep=1.5em,
    tabsize=6,
	belowskip=0em
}

\lstset{language=[mips]Assembler}
\lstset{style=mystyle}

\title{BPOS: Excercises for week 8}
\author{student: Dimitri Tabatadze (group: G3)}

\begin{document}
    \maketitle
	\pagebreak

    \section*{Solutions}

    \begin{enumerate}
        \item {
			\begin{enumerate}
				\item {
					\begin{displaymath}
						\begin{aligned}
							ba(a, c^0) &= sbase \\
							ba(b, c^0) &= sbase + 4 \\
							va(a, c^1) &= 256 \\
							d^0.m(lv(a, c)) &= enc(va(a, c^1), int) = enc(256, int) \\
							d^1.gpr(1) &= enc(va(a, c^1), int) = enc(256, int) \\
							d^2.gpr(2) &= ba(b, c^1) = sbase + 4 = d^1.gpr(28) + 4 && \text{\footnotesize without affecting register 1.} \\
							d^3.gpr(1) &= 0^4d^2.gpr(1)[31:4] = enc(16, int) && \text{\footnotesize code line number 11.} \\
							d.m(d^3.gpr(2)) &= d^3.gpr(1) = enc(16, int) && \text{\footnotesize $(d, c)$ is the desired configuration.} \\
							enc(va(b, c), int) &= d.m(d^3.gpr(2)) = enc(16, int) \\
							\implies va(b, c) &= 16 
						\end{aligned}
					\end{displaymath}
				}
				\item {
					no, since register 1 might get used by the statement
					\begin{lstlisting}
	gpr(2) = b&
\end{lstlisting}
					and the value 256 (which later becomes 16 because of the \verb|srl| instruction in the \verb|asm| block) would get lost. 
				}
			\end{enumerate}
		}
        \item {
			\begin{displaymath}
				\begin{aligned}
					\langle Reg \rangle \rightarrow\ &  {\verb|0|\ |\ \verb|1|\ |\ \dots|\ \verb|30|\ |\ \verb|31|} \\
					\langle RegS\rangle \rightarrow\ & \langle Reg\rangle\ |\ \langle Reg\rangle\verb|,|\langle RegS\rangle & \text{we could use \texttt{;} instead of \texttt{,}} \\
					\langle St\rangle \rightarrow\ & \verb|gpr(|\langle Reg\rangle\verb|) = |\langle E\rangle \verb| {|\langle RegS\rangle\verb|}| \ | & \text{Adding new productions}\\
					& \langle id\rangle\verb| = gpr(|\langle Reg\rangle\verb|) {|\langle RegS\rangle\verb|}|
				\end{aligned}
			\end{displaymath}
		}
        \item {
			From Lemma 93 I use only
			\begin{displaymath}
				d^k.gpr(j_5) = 
				\begin{cases}
					ba(va(e,c),c), & R(i) = 1 \land pointer(t) \land va(e,c)\neq null, \\
					enc(va(e,c),t), & R(i) = 1 \land t \in ET
				\end{cases}
			\end{displaymath}
			since $R(n') = 1$ is given where $n'$ is the node of the expression $e$ in the derivation tree. Let $j'$ be the pebble with the value of $n'$.
			\begin{displaymath}
				d''.gpr(j') = 
				\begin{cases}
					ba(va(e,c),c), & R(n') = 1 \land pointer(t) \land va(e,c)\neq null, \\
					enc(va(e,c),t), & R(n') = 1 \land t \in ET
				\end{cases}
			\end{displaymath}
			Since the last instruction is {\verb|addi j j' 0|}, $d''.gpr(j) = d''.gpr(j')$, so
			\begin{displaymath}
				d''.gpr(j) = 
				\begin{cases}
					ba(va(e,c),c), & pointer(t) \land va(e,c)\neq null, \\
					enc(va(e,c),t), & t \in ET
				\end{cases}
			\end{displaymath}
			
			Lemma 94:
			\begin{displaymath}
				d \rightarrow^*_{code(n,J)} d' \land j \in J \rightarrow d.gpr(j_5) = d'.gpr(j_5)
			\end{displaymath}
			where
			\begin{displaymath}
				J \subset [1:27]
			\end{displaymath}

			\pagebreak
			
			Let's prove that $i \in J \rightarrow d.gpr(i) = d''.gpr(i)$. $d \rightarrow_{code(n,J)} d''$ where $code(n, J)$ is
			\begin{lstlisting}[numbers=none,language=]
code(n', J)
addi j j' 0
\end{lstlisting}
			It's obvious that $j \not\in J$ and given that $code(n', J)$ won't affect $J$ (from Lemma 94) and the last instruction of $code(n, J)$, \ \verb|addi j j' 0| \ won't change anything but the register $j$ so the set $J$ will remain unchanged.

			Lemma 127 suggests that the registers $i \in [28:31]$ always remain unchanged. However the syntax for C+A doesn't forbid us from accessing the those registers.
		}
        \item \begin{enumerate}
			\item {\hspace*{1em}
				\begin{minipage}[t]{0.5\linewidth}
					\begin{lstlisting}[language=c,tabsize=2]
uint x;

int main() {
	x = 0;
	gpr(1) = x&;
	while (x < 10) {
		asm(
			addi 1 1 1
		)
	};
	return 0
}
\end{lstlisting}
				\end{minipage}
			}
			\item {Only 1 step is needed. After one step, the whole while block will be loaded --- the start of the node of the while loop will be the 2nd next step}
			\item {\hspace*{1em}
				\begin{minipage}[t]{0.5\linewidth}
					\begin{lstlisting}[language=c,tabsize=2]
int x;

int main() {
	x = 0;
	gpr(1) = x&;
	while (x < 10) {
		if (x < 9) {
			asm(
				addi 1 1 1
			)
		} else {
			x = 0
		}
	};
	return 0
}
\end{lstlisting}
				\end{minipage}
			}
		\end{enumerate}
    \end{enumerate}
\end{document}