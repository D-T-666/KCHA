\documentclass{article}

\usepackage[margin=3cm]{geometry}
\usepackage{amsmath}
\usepackage{amssymb}
\usepackage{enumitem}

\newcommand*{\mat}[1]{\begin{pmatrix}#1\end{pmatrix}}
\DeclareMathOperator{\adj}{adj}

\author{Dimitri Tabatadze}
\title{Homework 1}

\begin{document}
    \maketitle

    \begin{enumerate}
        \item {
            We know that the eigenvalues of a matrix are the solutions to the characteristic equation of the matrix. So a matrix would have no real eigenvalues if the characteristic polynomial has no solutions.
            \begin{displaymath}
                \begin{aligned}
                    \det(A-\lambda I) &= 0 & \implies \\
                    (2 - \lambda)(1 - \lambda) + a &= 0 & \implies \\
                    \lambda^2 -3\lambda + 2 + a &= 0
                \end{aligned}
            \end{displaymath}
            we now know that
            \begin{displaymath}
                \begin{aligned}
                    (-3)^2 - 8 - 4a &< 0 & \implies \\
                    4a &> 1 & \implies \\
                    a &> \frac{1}{4}
                \end{aligned}
            \end{displaymath}
        }
        \item {
            \begin{enumerate}
                \item {
                    \begin{displaymath}
                        \begin{cases}
                            3x-4y=-7\\
                            -6x+8y=14
                        \end{cases}
                        \implies
                        \begin{cases}
                            3x-4y=-7\\
                            0x+0y=0
                        \end{cases}
                        \implies
                        \begin{cases}
                            x = \frac{4t - 7}{3} \\
                            y = t
                        \end{cases}
                    \end{displaymath}
                }
                \item {
                    \begin{gather*}
                        \begin{cases}
                            -x+2y-4z = 8 \\
                            3y+8z = -4 \\
                            -7x+y+2z = 1
                        \end{cases}
                        \implies
                        \begin{cases}
                            -x+2y-4z = 8 \\
                            y+\frac{8}{3}z = -\frac{4}{3} \\
                            -13y-26z = -55 \\
                        \end{cases}
                        \implies\\
                        \begin{cases}
                            -x-\frac{28}{3}z = \frac{32}{3} \\
                            y+\frac{8}{3}z = -\frac{4}{3} \\
                            z = -\frac{217}{26} \\
                        \end{cases}
                        \implies
                        \begin{cases}
                            -x = \frac{32}{3} - \frac{28}{3}\frac{217}{26} \\
                            y = -\frac{4}{3} + \frac{8}{3}\frac{217}{26} \\
                            z = -\frac{217}{26} \\
                        \end{cases}
                        \implies
                        \begin{cases}
                            x = \frac{874}{13} \\
                            y = \frac{272}{13} \\
                            z = -\frac{217}{26} \\
                        \end{cases}
                    \end{gather*}
                }
            \end{enumerate}
        }
        \item {
            Let \(\alpha\) be the smallest angle and \(\beta\) be the largest.
            \begin{gather*}
                \begin{cases}
                    \alpha = \frac{1}{2}\beta + 10^\circ \\
                    180^\circ - \alpha - \beta = \alpha + 12^\circ
                \end{cases}
                \implies
                \begin{cases}
                    \alpha - \frac{1}{2}\beta = 10^\circ \\
                    2\alpha + \beta = 168^\circ
                \end{cases}
                \implies
                \begin{cases}
                    \alpha = 57^\circ \\
                    \beta = 94^\circ
                \end{cases}
            \end{gather*}
        }
        \item {
            \begin{itemize}
                \item {
                    \begin{displaymath}
                        \left\|\mat{-5 \\ 4 \\ 5}\right\|_1 = |-5|+|4|+|5| = 14
                    \end{displaymath}
                } 
                \item {
                    \begin{displaymath}
                        \left\|\mat{-5\\4\\5}\right\|_2 = \left((-5)^2+4^2+5^2\right)^\frac{1}{2} = \sqrt{66}
                    \end{displaymath}
                }
                \item {
                    \begin{displaymath}
                        \left\|\mat{-5\\4\\5}\right\|_3 = \left((-5)^3+4^3+5^3\right)^\frac{1}{3} = \sqrt[3]{64} = 4
                    \end{displaymath}
                }
                \item {
                    \begin{displaymath}
                        \left\|\mat{-5\\4\\5}\right\|_\infty = \max\left\{|-5|,|4|,|5|\right\} = 5
                    \end{displaymath}
                }
                \item {
                    \begin{displaymath}
                        \left\|\mat{-5\\4\\5}\right\|_\infty = \max\left\{|-5|,|4|,|5|\right\} = 5
                    \end{displaymath}
                }
                \item {
                    \begin{displaymath}
                        \left\|\mat{-5\\4\\5}\right\|_A = \sqrt{A\mat{-5\\4\\5}\cdot\mat{-5 & 4 & 5}} = \sqrt{\mat{-6 & 8 & 9}\cdot\mat{-5 & 4 & 5}} = \sqrt{107}
                    \end{displaymath}
                }
            \end{itemize}
        }
        \item {
            We need to show all three properties of vector norms
            \begin{enumerate}[label={\text{property }\arabic*}]
                \item {
                    \begin{displaymath}
                        \sum_{k=1}^n\left|\sum_{i=1}^k x_i\right| \geq 0 \impliedby
                        \left|\sum_{i=1}^k x_i\right| \geq 0
                    \end{displaymath}

                    \begin{displaymath}
                        \sum_{k=1}^n\left|\sum_{i=1}^k x_i\right| = 0 \Longleftrightarrow
                        x_i = 0 \ \forall i
                    \end{displaymath}
                }
                \item {
                    \begin{displaymath}
                        \sum_{k=1}^n\left|\sum_{i=1}^k \alpha x_i\right| = 
                        \sum_{k=1}^n\left|\alpha\cdot\sum_{i=1}^k x_i\right| = 
                        \sum_{k=1}^n|\alpha|\cdot\left|\sum_{i=1}^k x_i\right| = 
                        |\alpha|\cdot\sum_{k=1}^n\left|\sum_{i=1}^k x_i\right|
                    \end{displaymath}
                }
                \item {
                    \begin{gather*}
                        \sum_{k=1}^n\left|\sum_{i=1}^k(x_i+y_i)\right| = 
                        \sum_{k=1}^n\left|\sum_{i=1}^k x_i+ \sum_{i=1}^ky_i\right| \\ \leq \\
                        \sum_{k=1}^n\left(\left|\sum_{i=1}^k x_i\right| + \left|\sum_{i=1}^ky_i\right|\right) =
                        \sum_{k=1}^n\left|\sum_{i=1}^k x_i\right| + \sum_{k=1}^n\left|\sum_{i=1}^ky_i\right|
                    \end{gather*}
                }
            \end{enumerate}
        }
        \item {
            The problem can be rewriten as:
            \begin{displaymath}
                5|x_1| + |x_2| = 1 \implies \begin{cases}
                    x_1 = \pm\frac{1 - |x_2|}{5} \\
                    x_2 = \pm(1 - 5|x_2|)
                \end{cases}
            \end{displaymath}
        }
        \item {
            \(\mat{0 & 9 \\ 1 & 0}\mat{0 & 1 \\ 9 & 0} = \mat{81 & 0 \\ 0 & 1}\) has two eigenvalue, 
            \(\lambda = 1, 81\) so the spectral radius would be \(\rho(A) = 81\).
            \begin{itemize}
                \item {
                    \begin{displaymath}
                        \|A\|_1 = \max\{1, 9\} = 9
                    \end{displaymath}
                }
                \item {
                    \begin{displaymath}
                        \|A\|_2 = \sqrt{\rho(A^TA)} = 9
                    \end{displaymath}
                }
                \item {
                    \begin{displaymath}
                        \|A\|_\infty = 9
                    \end{displaymath}
                }
            \end{itemize}
        }
        \item {
            They \textit{really} are norms, because they are induced by their respective vector norms. 
        }
        \item {
            \begin{enumerate}
                \item {
                    First, we find the inverse of the matrix.
                    \begin{displaymath}
                        A^{-1} = \frac{1}{\det(A)}\mat{1 & -1 \\ -2 & 1} = \mat{-1 & 1 \\ 2 & -1}
                    \end{displaymath}
                    now we can calculate the condition number.
                    \begin{displaymath}
                        K(A)=\|A\|_1\|A^{-1}\|_1 = 3\cdot1 = 3
                    \end{displaymath}
                }
                \item {
                    First, we find the inverse of the matrix
                    \begin{displaymath}
                        A^{-1} = \frac{1}{\det(A)}\mat{8 & 0 & 0 \\ 0 & -6 & 0 \\ 0 & 0 & -12} = \mat{-1/3 & 0 & 0 \\ 0 & -1/4 & 0 \\ 0 & 0 & -1/2}
                    \end{displaymath}
                    now we can calculate the condition number.
                    \begin{displaymath}
                        K(A)=\|A\|_\infty\|A^{-1}\|_\infty = 4\cdot\frac{1}{2} = 2
                    \end{displaymath}
                }
            \end{enumerate}
        }
    \end{enumerate}
\end{document}