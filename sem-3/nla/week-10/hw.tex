\documentclass{article}


\usepackage{soul}
\usepackage[dark,christmas]{../../classes/dim}

\usepackage[normalem]{ulem}
\usepackage{mathtools}

\DeclareMathOperator{\adj}{adj}

\begin{document}
    \header{Homework}
    {Numerical Linear Algebra}

    \begin{enumerate}
        \item {Sherman-Morrison-Woodbury formula
            \begin{displaymath}
                (A+UCV)^{-1} = A^{-1}-A^{-1}U(C^{-1}+VA^{-1}U)^{-1}VA^{-1}
            \end{displaymath}
            Where \(A\in\R{n\times n}, U\in\R{n\times k}, C\in\R{k\times k}, V\in\R{k\times n}\).
            \begin{proof}
                \begin{displaymath}
                    \begin{aligned}
                        &\phantom{\mbox{}=\mbox{}} (A+UCV)(A^{-1}-A^{-1}U(C^{-1}+VA^{-1}U)^{-1}VA^{-1}) \\
                        &= I - \CR{U(C^{-1}+VA^{-1}U)^{-1}VA^{-1}} + \CG{UCVA^{-1}} - \CB{UCVA^{-1}U(C^{-1}+VA^{-1}U)^{-1}VA^{-1}} \\
                        &= I + \CG{UCVA^{-1}} - \CR{U(C^{-1}+VA^{-1}U)^{-1}VA^{-1}} - \CB{UCVA^{-1}U(C^{-1}+VA^{-1}U)^{-1}VA^{-1}} \\
                        &= I + \CG{UCVA^{-1}} - (U(C^{-1}+VA^{-1}U)^{-1} + UCVA^{-1}U(C^{-1}+VA^{-1}U)^{-1})VA^{-1} \\
                        &= I + \CG{UCVA^{-1}} - (U + UCVA^{-1}U)(C^{-1}+VA^{-1}U)^{-1}VA^{-1} \\
                        &= I + \CG{UCVA^{-1}} - UC(C^{-1} + VA^{-1}U)(C^{-1}+VA^{-1}U)^{-1}VA^{-1} \\
                        &= I + \CG{UCVA^{-1}} - UCVA^{-1} \\
                        &= I \\
                    \end{aligned}
                \end{displaymath}
                (source: \url{https://en.wikipedia.org/wiki/Woodbury_matrix_identity#Direct_proof})
            \end{proof}
        }
        \item {
            \begin{enumerate}
                \item {
                    \begin{displaymath}
                        A = \mat{1 & 2 \\ 2 & 8} = \mat{1 & 0 \\ 2 & 2}\mat{1 & 2 \\ 0 & 2} = HH^T
                    \end{displaymath}
                }
                \item {
                    \begin{displaymath}
                        A = \mat{4 & -2 & 0 \\ -2 & 2 & -3 \\ 0 & -3 & 10} = \mat{2 & 0 & 0 \\ -1 & 1 & 0 \\ 0 & -3 & 1}\mat{2 & -1 & 0 \\ 0 & 1 & -3 \\ 0 & 0 & 1} = HH^T
                    \end{displaymath}
                }
            \end{enumerate}
        }
        \item {
            \begin{enumerate}
                \item {
                    If we try to decompose the matrix we get
                    \begin{displaymath}
                        \begin{aligned}
                            A = \mat{1 & 2 \\ 2 & 2} &=
                            \mat{a & 0 \\ b & c}\mat{a & b \\ 0 & c} \\
                            &=
                            \mat{a^2 & ab \\ ab & b^2 + c^2}\\
                            &\Rightarrow
                            \begin{cases}
                                a^2 = 1 \\
                                ab = 2 \\
                                b^2 + c^2 = 2
                            \end{cases}
                            \Rightarrow
                            \begin{cases}
                                a = 1 \\
                                b = 2 \\
                                c^2 = -2.
                            \end{cases}
                        \end{aligned}
                    \end{displaymath}
                    But it is also clear that Cholesky decomposition would fail, since the matrix is not positive semi-definite.
                }
                \item {
                    The matrix is not positive semi-definite, thus it won't have a Cholesky decomposition.
                }
            \end{enumerate}
        }
        \item {
            \begin{displaymath}
                A = 
                \mat{10 & 3 & 0 \\ 4 & 9 & 4 \\ 0 & 3 & 5},\
                Ax = A
                \mat{x_1 \\ x_2 \\ x_3}
                = b =
                \mat{b_1 \\ b_2 \\ b_3}
            \end{displaymath}
            \begin{enumerate}
                \item \begin{enumerate}
                    \item {
                        \begin{displaymath}
                            x_1^{(k+1)} = \frac{b_1 - 3x_2^{(k)}}{10} \qquad
                            x_2^{(k+1)} = \frac{b_2 - 4x_1^{(k)} - 4x_3^{(k)}}{9} \qquad
                            x_3^{(k+1)} = \frac{b_3 - 3x_2^{(k)}}{5}
                        \end{displaymath}
                    }
                    \item {
                        \begin{displaymath}
                            x^{(k+1)} = -D^{-1}(L+U)x^{(k)} + D^{-1}b
                        \end{displaymath}
                        where
                        \begin{displaymath}
                            L = \mat{0 & 0 & 0 \\ 4 & 0 & 0 \\ 0 & 3 & 0},\quad
                            D = \mat{10 & 0 & 0 \\ 0 & 9 & 0 \\ 0 & 0 & 5},\quad
                            U = \mat{0 & 3 & 0 \\ 0 & 0 & 4 \\ 0 & 0 & 0}
                        \end{displaymath}
                        \begin{displaymath}
                            \begin{aligned}
                                x^{(k+1)} &= 
                                -\mat{1/10 & 0 & 0 \\ 0 & 1/9 & 0 \\ 0 & 0 & 1/5}
                                \left(\mat{0 & 0 & 0 \\ 4 & 0 & 0 \\ 0 & 3 & 0}
                                +\mat{0 & 3 & 0 \\ 0 & 0 & 4 \\ 0 & 0 & 0}\right)
                                x^{(k)} + 
                                \mat{1/10 & 0 & 0 \\ 0 & 1/9 & 0 \\ 0 & 0 & 1/5}
                                b \\
                                &= 
                                -\mat{1/10 & 0 & 0 \\ 0 & 1/9 & 0 \\ 0 & 0 & 1/5}
                                \mat{0 & 3 & 0 \\ 4 & 0 & 4 \\ 0 & 3 & 0}
                                x^{(k)} + 
                                \mat{1/10 & 0 & 0 \\ 0 & 1/9 & 0 \\ 0 & 0 & 1/5}
                                b \\
                                &= -\mat{0 & 3/10 & 0 \\ 4/9 & 0 & 4/9 \\ 0 & 3/5 & 0}
                                x^{(k)} + 
                                \mat{1/10 & 0 & 0 \\ 0 & 1/9 & 0 \\ 0 & 0 & 1/5}
                                b
                            \end{aligned}
                        \end{displaymath}
                    }
                \end{enumerate}
                \item {\begin{enumerate}
                    \item {\begin{displaymath}
                        x_1^{(k+1)} = \frac{b_1 - 3x_2^{(k)}}{10},\qquad
                        x_2^{(k+1)} = \frac{b_2 - 4x_1^{(k+1)} - 4x_3^{(k)}}{9} \qquad
                        x_3^{(k+1)} = \frac{b_3 - 3x_2^{(k+1)}}{5}
                    \end{displaymath}}
                    \item {
                        \begin{displaymath}
                            x^{(k+1)} = (D+L)^{-1}(b-Ux^{(k)})
                        \end{displaymath}
                        where
                        \begin{displaymath}
                            L = \mat{0 & 0 & 0 \\ 4 & 0 & 0 \\ 0 & 3 & 0},\quad
                            D = \mat{10 & 0 & 0 \\ 0 & 9 & 0 \\ 0 & 0 & 5},\quad
                            U = \mat{0 & 3 & 0 \\ 0 & 0 & 4 \\ 0 & 0 & 0}
                        \end{displaymath}
                        \begin{displaymath}
                            \begin{aligned}
                                x^{(k+1)} &= 
                                \left(
                                    \mat{10 & 0 & 0 \\ 0 & 9 & 0 \\ 0 & 0 & 5}
                                    +\mat{0 & 0 & 0 \\ 4 & 0 & 0 \\ 0 & 3 & 0}
                                \right)^{-1}
                                \left(
                                    b
                                    -\mat{0 & 3 & 0 \\ 0 & 0 & 4 \\ 0 & 0 & 0}
                                    x^{(k)}
                                \right) \\
                                &= 
                                \mat{10 & 0 & 0 \\ 4 & 9 & 0 \\ 0 & 3 & 5}^{-1}
                                \left(
                                    b
                                    -\mat{0 & 3 & 0 \\ 0 & 0 & 4 \\ 0 & 0 & 0}
                                    x^{(k)}
                                \right) \\
                            \end{aligned}
                        \end{displaymath}
                    }
                \end{enumerate}}
                \item {\begin{enumerate}
                    \item sufficient condition \(\|A\|_1 = 15 \not< 1\) --- not satisfied.
                    \item necessary condition \(\rho(A) = \max\{|\lambda_1|, |\lambda_2|, |\lambda_3|\} \approx 2.531 \not< 1\) --- not satisfied.
                \end{enumerate}}
                \item {
                    We know that
                    \begin{displaymath}
                        \|e^{(k+1)}\| 
                        = \|B^{k+1}e^{(0)}\| 
                        \leq \|B^{k+1}\|\cdot\|e^{(0)}\| 
                        \leq \|B\|^{k+1}\cdot\|e^{(0)}\|
                    \end{displaymath}
                    We can calculate \(\|B\|\) to be \CG{\(\|B\|_1 = 9/10\)} for the \CG{Jacobi method} 
                    and \CB{\(\|B\|_1 = 1/5\)} for the \CB{Gauss-Seidel method}. We get that for \CG{Jacobi
                    method} we would need \CG{\(\lceil\log_{9/10}(1/10)\rceil = \left\lceil\frac{\log_{10}1 - 1}{\log_{10}9 - 1}\right\rceil
                    = \left\lceil\frac{1}{1 - \log_{10}9}\right\rceil = 22\)} iterations. For \CB{Gauss-Seidel method}
                    we would need \CB{\(\lceil\log_{1/5}(1/10)\rceil = 2\)}. Allthough, from the previous 
                    sub-task we know that neither of the methods will converge.
                }
            \end{enumerate}
        }
        \item {
            \begin{displaymath}
                \mat{3 & 1 & 1 \\ 1 & 3 & 1 \\ 1 & 1 & 3}
                \mat{u \\ v \\ w}
                =
                \mat{6 \\ 3 \\ 5},\quad
                \omega = 1.5,\quad
                u^{(0)} = v^{(0)} = w^{(0)} = 0
            \end{displaymath}
            \begin{enumerate}[label={\#\arabic*}]
                \item {
                    \begin{displaymath}
                        \begin{aligned}
                            u^{(1)} &= \frac{1.5}{3} 6 = 3 \\
                            v^{(1)} &= \frac{1.5}{3} (3 - 3) = 0 \\
                            w^{(1)} &= \frac{1.5}{3} (5 - 3) = 1
                        \end{aligned}
                    \end{displaymath}
                }
                \item {
                    \begin{displaymath}
                        \begin{aligned}
                            u^{(2)} &= \frac{1.5}{3} (6 - 1) + (1 - 1.5)3 = 1 \\
                            v^{(2)} &= \frac{1.5}{3} (3 - 1 - 1) = 0.5 \\
                            w^{(2)} &= \frac{1.5}{3} (5 - 1.5) + (1 - 1.5) = 1.25
                        \end{aligned}
                    \end{displaymath}
                }
            \end{enumerate}
            Result:
            \begin{displaymath}
                \mat{u^{(2)} \\ v^{(2)} \\ w^{(2)}}  =
                \mat{1 \\ 0.5 \\ 1.25}
            \end{displaymath}
        }
        \item {
            \begin{displaymath}
                A = 
                \mat{5 & 3 \\ 1 & 4},\quad
                P = 
                \mat{5 & 3 \\ 3 & 5}
            \end{displaymath}
        }
    \end{enumerate}
\end{document}