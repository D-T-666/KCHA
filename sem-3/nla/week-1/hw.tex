\documentclass{article}

\usepackage[margin=3cm]{geometry}
\usepackage{amsmath}

\newcommand*{\mat}[1]{\begin{pmatrix}#1\end{pmatrix}}
\DeclareMathOperator{\adj}{adj}

\author{Dimitri Tabatadze}
\title{Homework 1}

\begin{document}
    \maketitle

    \begin{enumerate}
        \item { step by step solution:
            \begin{displaymath}
                \begin{aligned}
                    & u \cdot v = 0, \|u\| = 15 \\ \implies &
                    \begin{cases}
                        3 \cdot x + 4 \cdot y = 0 \\
                        \sqrt{x^2 + y^2} = 15
                    \end{cases} \\ \implies &
                    \begin{cases}
                        3 \cdot x = -4 \cdot y \\
                        x^2 + y^2 = 15^2
                    \end{cases} \\ \implies &
                    \begin{cases}
                        x = -\frac{4}{3} \cdot y \\
                        x^2 + y^2 = 15^2
                    \end{cases} \\ \implies &
                    \begin{cases}
                        x = -\frac{4}{3} \cdot y \\
                        \left(-\frac{4}{3} \cdot y\right)^2 + y^2 = 15^2
                    \end{cases} \\ \implies &
                    \begin{cases}
                        x = -\frac{4}{3} \cdot y \\
                        y^2\left(\frac{16}{9} + 1\right) = 15^2
                    \end{cases} \\ \implies &
                    \begin{cases}
                        y = \sqrt{15^2\cdot\frac{9}{25}} = 9 \\
                        x = -12
                    \end{cases} \\ \implies &
                    u = (-12, 9)
                \end{aligned}
            \end{displaymath}
            now it's clear that
            \begin{displaymath}
                -12\cdot3 + 9\cdot4 = 0
            \end{displaymath}
            and
            \begin{displaymath}
                \sqrt{(-12)^2 + 9^2} = 15
            \end{displaymath}
        }
        \item {
            let's precalculate the cross product
            \begin{displaymath}
                \vec{u}\times\vec{v} = (1\cdot0-(-2)\cdot(-1), (-1)\cdot0-(-2)\cdot2, (-1)\cdot(-1)-2\cdot1) = (-2,4,-1)
            \end{displaymath}
            and the dot product
            \begin{displaymath}
                \vec{u}\cdot\vec{v} = (-1)\cdot2+1\cdot(-1)+(-2)\cdot0 = -3
            \end{displaymath}
            and the magnitudes
            \begin{displaymath}
                \begin{aligned}
                    \|\vec{u}\| &= \sqrt{(-1)^2+1^2+(-2)^2} = \sqrt{6} \\
                    \|\vec{v}\| &= \sqrt{2^2+(-1)^2+0^2} = \sqrt{5} \\
                    \|\vec{u}\times\vec{v}\| &= \sqrt{(-2)^2+4^2+(-1)^2} = \sqrt{21} \\
                \end{aligned}
            \end{displaymath}
            so now
            \begin{displaymath}
                \begin{aligned}
                    &&\|\vec{u}\times\vec{v}\|^2 &= \|\vec{u}\|^2\|\vec{v}\|^2 - (\vec{u}\cdot\vec{v})^2  \\ \implies &
                    &21 &= 6\cdot5 - (-3)^2  \\ \implies &
                    &21 &= 30 - 9  \\ \implies &
                    &21 &= 21
                \end{aligned}
            \end{displaymath}
        }
        \item {
            \begin{enumerate}
                \item {
                    \begin{displaymath}
                        \begin{aligned}
                            \vec{AB} &= B - A = (0 - 2, 1 - (-3), 2 - 4) = (-2, 4, -2) \\
                            \vec{AC} &= C - A = ((-1) - 2, 2 - (-3), 0 - 4) = (-3, 5, -4) \\
                            \vec{AB}\times\vec{AC} &= (4\cdot(-4)-5\cdot(-2), (-2)\cdot(-4)-(-3)\cdot(-2), (-3)\cdot4-(-2)\cdot5) \\
                            &= (-6, 2, -2) \\
                            \|\vec{AB}\times\vec{AC}\| &= \sqrt{(-6)^2+2^2+(-2)^2} = 2\sqrt{11}
                        \end{aligned}
                    \end{displaymath}
                    so the area of parallelogram \(ABCD\) with adjacent sides \(\vec{AB}\) and \(\vec{AC}\) is \(2\sqrt{11}\).
                }
                \item {
                    the area of the triangle \(ABC\) would be half of that of the parallelogram \(ABCD\) which is \(\sqrt{11}\).
                }
            \end{enumerate}
        }
        \item { First, let's find the vectors
            \begin{displaymath}
                \begin{aligned}
                    \vec{AB} &= (6-4, 5-1, -2-0) = (2, 4, -2)\\
                    \vec{AC} &= (5-4, 3-1, -1-0) = (1, 2, -1)\\
                \end{aligned}
            \end{displaymath}
            now it's clear that \(2\vec{AC} = \vec{AB}\)
        }
        \item {
            the angle between two vectors \(\vec{u}\) and \(\vec{v}\) is equal to 
            \begin{displaymath}
                \cos^{-1}\left(\frac{\vec{u}\cdot\vec{v}}{\|\vec{u}\|\cdot\|\vec{v}\|}\right)
            \end{displaymath}
            so the angle between \(\vec{OP}\) and \(\vec{OQ}\) will be
            \begin{displaymath}
                \begin{aligned}
                    &\cos^{-1}\left(\frac{\vec{OP}\cdot\vec{OQ}}{\|\vec{OP}\|\cdot\|\vec{OQ}\|}\right) \\
                    = &\cos^{-1}\left(\frac{3\cdot1+7\cdot1+(-2)\cdot(-3)}{\sqrt{3^2+7^2+(-2)^2}\cdot\sqrt{1^2+1^2+(-3)^2}}\right) \\
                    = &\cos^{-1}\left(\frac{3+7+6}{\sqrt{9+49+4}\cdot\sqrt{1+1+9}}\right) \\
                    = &\cos^{-1}\left(\frac{16}{\sqrt{62}\cdot\sqrt{11}}\right) \\
                \end{aligned}
            \end{displaymath}
        }
        \item { step by step solution:
            \begin{displaymath}
                \begin{aligned}
                    &\mat{3 & 0 \\ -1 & 2}\mat{3 & -1 \\ 0 & 2} - 2\mat{3 & -1 \\ 0 & 2} \\
                    =&\mat{3\cdot3+0\cdot0 & 3\cdot(-1)+0\cdot2 \\ (-1)\cdot3+2\cdot0 & (-1)\cdot(-1)+2\cdot2} - \mat{2\cdot3 & 2\cdot(-1) \\ 2\cdot0 & 2\cdot2} \\
                    =&\mat{9 & -3 \\ -3 & 5} - \mat{6 & -2 \\ 0 & 4} \\
                    =&\mat{9 - 6 & -3 - (-2) \\ -3 - 0 & 5 - 4} \\
                    =&\mat{3 & -1 \\ -3 & 1} \\
                \end{aligned}
            \end{displaymath}
        }
        \item {
            Let \(X = \mat{a & b \\ c & d}\), then \(XA = B\) can be writen as
            \begin{displaymath}
                \mat{a & b \\ c & d}\mat{2 & 1 \\ -4 & -3} = \mat{2 & 2 \\ 6 & 4}
            \end{displaymath}
            which, by definition, is the following system
            \begin{displaymath}
                \begin{cases}
                    2\cdot a + (-4)\cdot b = 2 \\
                    1\cdot a + (-3)\cdot b = 2 \\
                    2\cdot c + (-4)\cdot d = 6 \\
                    1\cdot c + (-3)\cdot d = 4 \\
                \end{cases}
                \implies
                \begin{cases}
                    a - 2 b = 1 \\
                    a - 3 b = 2 \\
                    c - 2 d = 3 \\
                    c - 3 d = 4 \\
                \end{cases}
                \implies
                \begin{cases}
                    a = -1 \\
                    b = -1 \\
                    c = 1 \\
                    d = -1 \\
                \end{cases}
            \end{displaymath}
            so \(X=\mat{-1 & -1 \\ 1 & -1}\)
        }
        \item {
            The cofactor \(C_{ij} = (-1)^{i+j}M_{ij}\) where \(M_{ij}\) is the minor of the matrix.

            \begin{enumerate}
                \item {
                    Let \(A = \mat{1 & 1 & 2 \\ 1 & -1 & 1 \\ 0 & -1 & 3}\).

                    The determinant \(\det(A)\) can be calculated in the following way:
                    \begin{displaymath}
                        \begin{aligned}
                            \det(A) &= 
                            (1 \cdot (-1) \cdot 3) +
                            (1 \cdot 1 \cdot 0) +
                            (1 \cdot (-1) \cdot 2) -
                            (2 \cdot (-1) \cdot 0) -
                            (1 \cdot (-1) \cdot 1) -
                            (1 \cdot 1 \cdot 3) \\
                            &= -3 + 0 - 2 - 0 + 1 - 3 \\
                            &= -7
                        \end{aligned}
                    \end{displaymath}
                    Now to calculate \(\adj(A)\), we first need to calculate all 9 cofactors
                    \begin{displaymath}
                        \begin{aligned}
                            C_{11} &= (-1)^{1+1} \cdot \left((-1)\cdot3 - 1\cdot(-1)\right) &=& -2 \\
                            C_{12} &= (-1)^{1+2} \cdot \left(1\cdot3 - 1\cdot0\right) &=& -2 \\
                            C_{13} &= (-1)^{1+3} \cdot \left(1\cdot(-1) - (-1)\cdot0\right) &=& -1 \\
                            C_{21} &= (-1)^{2+1} \cdot \left(1\cdot3 - 2\cdot(-1)\right) &=& -5 \\
                            C_{22} &= (-1)^{2+2} \cdot \left(1\cdot3 - 2\cdot2\right) &=& -1 \\
                            C_{23} &= (-1)^{2+3} \cdot \left(1\cdot(-1) - 1\cdot0\right) &=& 2 \\
                            C_{31} &= (-1)^{3+1} \cdot \left(1\cdot1 - 2\cdot(-1)\right) &=& 3 \\
                            C_{32} &= (-1)^{3+2} \cdot \left(1\cdot1 - 2\cdot1\right) &=& 1 \\
                            C_{33} &= (-1)^{3+3} \cdot \left(1\cdot(-1) - 1\cdot1\right) &=& -2 \\
                        \end{aligned}
                    \end{displaymath}
                    and we get that
                    \begin{displaymath}
                        \adj(A) = \mat{-2 & -5 & 3 \\ -2 & -1  & 1 \\ -1 & 2 & -2}
                    \end{displaymath}
                    now we can finaly find the inverse
                    \begin{displaymath}
                        A^{-1} = \frac{1}{\det(A)}\cdot\adj(A) = \frac{1}{-7}\cdot\mat{-2 & -5 & 3 \\ -2 & -1  & 1 \\ -1 & 2 & -2} = \mat{{2 / 7} & {5 / 7} & {-3 / 7} \\ {2 / 7} & {1 / 7} & {-1 / 7} \\ {1 / 7} & {-2 / 7} & {2 / 7}}
                    \end{displaymath}
                }
                \item {
                    Let \(B = \mat{1 & 2 & 3 \\ 3 & 1 & 2 \\ 2 & 3 & 1}\).

                    The determinant \(\det(B)\) can be calculated in the following way:
                    \begin{displaymath}
                        \begin{aligned}
                            \det(B) &= 
                            (1 \cdot 1 \cdot 1) +
                            (2 \cdot 2 \cdot 2) +
                            (3 \cdot 3 \cdot 3) -
                            (3 \cdot 1 \cdot 2) -
                            (2 \cdot 3 \cdot 1) -
                            (1 \cdot 2 \cdot 3) \\
                            &= 1 + 8 + 27 - 6 - 6 - 6 \\
                            &= 18
                        \end{aligned}
                    \end{displaymath}

                    Now to calculate \(\adj(B)\), we first need to calculate all 9 cofactors
                    \begin{displaymath}
                        \begin{aligned}
                            C_{11} &= (-1)^{1+1} \cdot \left(1\cdot1 - 2\cdot3\right) &=& -5 \\
                            C_{12} &= (-1)^{1+2} \cdot \left(3\cdot1 - 2\cdot2\right) &=& 1 \\
                            C_{13} &= (-1)^{1+3} \cdot \left(3\cdot3 - 1\cdot2\right) &=& 7 \\
                            C_{21} &= (-1)^{2+1} \cdot \left(2\cdot1 - 3\cdot3\right) &=& 7 \\
                            C_{22} &= (-1)^{2+2} \cdot \left(1\cdot1 - 3\cdot2\right) &=& -5 \\
                            C_{23} &= (-1)^{2+3} \cdot \left(1\cdot3 - 2\cdot2\right) &=& 1 \\
                            C_{31} &= (-1)^{3+1} \cdot \left(2\cdot2 - 3\cdot1\right) &=& 1 \\
                            C_{32} &= (-1)^{3+2} \cdot \left(1\cdot2 - 3\cdot3\right) &=& 7 \\
                            C_{33} &= (-1)^{3+3} \cdot \left(1\cdot1 - 2\cdot3\right) &=& -5 \\
                        \end{aligned}
                    \end{displaymath}
                    and we get that
                    \begin{displaymath}
                        \adj(A) = \mat{-5 & 7 & 1 \\ 1 & -5  & 7 \\ 7 & 1 & -5}
                    \end{displaymath}
                    now we can finaly find the inverse
                    \begin{displaymath}
                        A^{-1} = \frac{1}{\det(A)}\cdot\adj(A) = \frac{1}{18}\cdot\mat{-5 & 7 & 1 \\ 1 & -5  & 7 \\ 7 & 1 & -5} = \mat{-5/18 & 7/18 & 1/18 \\ 1/18 & -5/18 & 7/18 \\ 7/17 & 1/18 & -5/18}
                    \end{displaymath}
                }
            \end{enumerate}
        }
    \end{enumerate}
\end{document}