\documentclass{article}

\usepackage{soul}
\usepackage[dark,christmas,showauthor]{../../../classes/dim}

\usepackage[normalem]{ulem}
\usepackage{mathtools}

\usepackage{listings}
\lstset{
language=Python,
basicstyle=\ttfamily\footnotesize,
morekeywords={self},              % Add keywords here
keywordstyle=\color{r},
emph={MyClass,__init__},          % Custom highlighting
emphstyle=\color{r},    % Custom highlighting style
stringstyle=\color{g},
frame=tb,                         % Any extra options here
showstringspaces=false
}

\DeclareMathOperator{\adj}{adj}

\begin{document}
    \header{Homework}
    {Numerical Linear Algebra}

    \begin{tasks}
        \item {
            \begin{displaymath}
                A = \mat{2 & 1 \\ 1 & 2},\ 
                P = I
            \end{displaymath}

            The richardson method will go like this:

            \begin{displaymath}
                \begin{gathered}
                    \begin{aligned}
                        I\frac{x^{(k+1)}-x^{(k)}}{\tau} + Ax^{(k)} 
                        &= b \\
                        x^{(k+1)}-x^{(k)} 
                        &= \tau I^{-1}b - \tau I^{-1}Ax^{(k)} \\
                        x^{(k+1)}
                        &= (I - \tau I^{-1}A)x^{(k)} + \tau b
                    \end{aligned}
                    \\ \Downarrow \\
                    \begin{aligned}
                        \tau_\text{opt} 
                        &= \frac{2}{\lambda_\text{max}(I^{-1}A) + \lambda_\text{min}(I^{-1}A)} \\
                        &= \frac{2}{\lambda_\text{max}(A) + \lambda_\text{min}(A)} \\
                        &= \frac{2}{1 + 3} \\
                        &= \frac{1}{2}
                    \end{aligned}
                \end{gathered}
            \end{displaymath}
        }
        \item {
            I used the code from the task 3 to calculate the solution:
            \begin{displaymath}
                x \approx \mat{0.995 \\ 0.957 \\ 0.791}
            \end{displaymath}
        }
        \item {
            I think that this code is very much self-descriptive.
            \lstinputlisting[language=python]{sor.py}
        }
        \item {
            I picked \(x^{(0)}\) to be a vector with all ones. Using the code I wrote for the task 3, this is the solution I get:
            
            \begin{displaymath}
                x \approx \tiny \mat{
                    1.53872\dots \\ 0.73141\dots \\ 0.10797\dots \\ 0.17328\dots \\ 0.04055\dots \\ 0.08524\dots \\ 0.16644\dots \\ 0.12197\dots \\ 0.10125\dots \\ 0.09045\dots \\ 0.07202\dots \\ 0.07026\dots \\ 0.06875\dots \\ 0.06324\dots \\ 0.05971\dots \\ 0.05570\dots \\ 0.05187\dots \\ 0.04924\dots \\ 0.04677\dots \\ 0.04448\dots \\ 0.04246\dots \\ 0.04053\dots \\ 0.03876\dots \\ 0.03717\dots \\ 0.03570\dots \\ 0.03434\dots \\ 0.03309\dots \\ 0.03191\dots \\ 0.03082\dots \\ 0.02980\dots \\ 0.02885\dots \\ 0.02795\dots \\ 0.02711\dots \\ 0.02632\dots \\ 0.02557\dots \\ 0.02486\dots \\ 0.02419\dots \\ 0.02356\dots \\ 0.02296\dots \\ 0.02239\dots \\ 0.02184\dots \\ 0.02133\dots \\ 0.02083\dots \\ 0.02036\dots \\ 0.01991\dots \\ 0.01948\dots \\ 0.01906\dots \\ 0.01867\dots \\ 0.01829\dots \\ 0.01792\dots \\ 0.01757\dots \\ 0.01723\dots \\ 0.01691\dots \\ 0.01660\dots \\ 0.01630\dots \\ 0.01601\dots \\ 0.01573\dots \\ 0.01546\dots \\ 0.01519\dots \\ 0.01494\dots \\ 0.01470\dots \\ 0.01446\dots \\ 0.01423\dots \\ 0.01401\dots \\ 0.01380\dots \\ 0.01359\dots \\ 0.01338\dots \\ 0.01318\dots \\ 0.01297\dots \\ 0.01278\dots \\ 0.01270\dots \\ 0.01252\dots \\ 0.01237\dots \\ 0.01220\dots \\ 0.01129\dots \\ 0.01114\dots \\ 0.01217\dots \\ 0.01201\dots \\ 0.01542\dots \\ 0.01523\dots
                }
            \end{displaymath}
        }
    \end{tasks}
\end{document}