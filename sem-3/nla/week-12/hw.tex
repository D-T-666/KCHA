\documentclass{article}

\usepackage{soul}
\usepackage[dark,showauthor]{../../../classes/dim}

\usepackage[normalem]{ulem}
\usepackage{mathtools}

\usepackage{listings}
\lstset{
language=Python,
basicstyle=\ttfamily\footnotesize,
morekeywords={self},              % Add keywords here
keywordstyle=\color{r},
emph={MyClass,__init__},          % Custom highlighting
emphstyle=\color{r},    % Custom highlighting style
stringstyle=\color{g},
frame=tb,                         % Any extra options here
showstringspaces=false
}

\DeclareMathOperator{\adj}{adj}

\begin{document}
    \header{Homework}
    {Numerical Linear Algebra}

    \begin{tasks}
        \item The steps:
        \begin{enumerate}[label={(\arabic*)}]
            \item \begin{displaymath}
                \begin{gathered}
                    r_0 = b - Ax_0 = \mat{1 \\ 0 \\ 0} - \mat{0 \\ 0 \\ 0} = \mat{1 \\ 0 \\ 0} \\
                    \alpha_0 = \frac{\langle Ar_0, r_0\rangle}{\langle Ar_0, Ar_0 \rangle} 
                    = \frac{\left\langle \mat{1 \\ 0 \\ 1}, \mat{1 \\ 0 \\ 0}\right\rangle}{\left\langle \mat{1 \\ 0 \\ 1}, \mat{1 \\ 0 \\ 1}\right\rangle} 
                    = \frac{1}{2} \\
                    x_1 = x_0 + \alpha_0r_0 = \mat{1/2 \\ 0 \\ 0}
                \end{gathered}
            \end{displaymath}
            \item \begin{displaymath}
                \begin{gathered}
                    r_1 = b - Ax_1 = \mat{1 \\ 0 \\ 0} - \mat{1/2 \\ 0 \\ 1/2} = \mat{1/2 \\ 0 \\ -1/2} \\
                    \alpha_1 = \frac{\langle Ar_1, r_1\rangle}{\langle Ar_1, Ar_1 \rangle} 
                    = \frac{\left\langle \mat{1/2 \\ 0 \\ 0}, \mat{1/2 \\ 0 \\ -1/2}\right\rangle}{\left\langle \mat{1/2 \\ 0 \\ 0}, \mat{1/2 \\ 0 \\ 0}\right\rangle} 
                    = 1 \\
                    x_2 = x_1 + \alpha_1r_1 = \mat{1 \\ 0 \\ -1/2}
                \end{gathered}
            \end{displaymath}
            \item \begin{displaymath}
                \begin{gathered}
                    r_2 = b - Ax_2 = \mat{1 \\ 0 \\ 0} - \mat{1 \\ 0 \\ -1/2} = \mat{0 \\ 0 \\ 1/2} \\
                    \alpha_2 = \frac{\langle Ar_2, r_2\rangle}{\langle Ar_2, Ar_2 \rangle} 
                    = \frac{\left\langle \mat{0 \\ 0 \\ 1/2}, \mat{0 \\ 0 \\ 1/2}\right\rangle}{\left\langle \mat{0 \\ 0 \\ 1/2}, \mat{0 \\ 0 \\ 1/2}\right\rangle} 
                    = 1 \\
                    x_3 = x_2 + \alpha_2r_2 = \mat{1 \\ 0 \\ -1}
                \end{gathered}
            \end{displaymath}
            \item \begin{displaymath}
                \begin{gathered}
                    r_2 = b - Ax_3 = \mat{1 \\ 0 \\ 0} - \mat{1 \\ 0 \\ -1} = \mat{0 \\ 0 \\ 0} \\
                \end{gathered}
            \end{displaymath}
            stop.
        \end{enumerate}
        \item \begin{enumerate}
            \item \begin{displaymath}
                A 
                = \mat{1 & 2 \\ 1 & 1} 
                = \mat{\frac{1}{\sqrt{2}} & \frac{1}{2\sqrt{2}} \\ \frac{1}{\sqrt{2}} & -\frac{1}{2\sqrt{2}}}
                \mat{\sqrt{2} & \sqrt{2} \\ 0 & \sqrt{2}} 
                = QR
            \end{displaymath}
            \item \begin{displaymath} % 23/9 -50/9 -2/9  529+2500+4/81  3033/3^3  337/9 
                A 
                = \mat{2 & 3 \\ -2 & -6 \\ 1 & 0} 
                = \mat{\frac29 & \frac{23}{3\sqrt{337}} \\ -\frac29 & -\frac{50}{3\sqrt{337}} \\ \frac19 & -\frac{2}{3\sqrt{337}}}
                \mat{\sqrt{2} & 2 \\ 0 & \frac{\sqrt{337}}{3}} 
                = QR
            \end{displaymath}
        \end{enumerate}
        \item \begin{enumerate}
            \item \begin{displaymath}
                \begin{gathered}    
                    \mat{2 & 3 \\ -2 & -6 \\ 1 & 0} \mat{x_1 \\ x_2} = \mat{3 \\ -3 \\ 6} \\
                    R\mat{x_1 \\ x_2} = Q^T\mat{3 \\ -3 \\ 6} \\
                    \mat{\sqrt{2} & 2 \\ 0 & \frac{\sqrt{337}}{3}}\mat{x_1 \\ x_2}
                    = \mat{\frac29 & -\frac29 & \frac19  \\ \frac{23}{3\sqrt{337}} & -\frac{50}{3\sqrt{337}} & -\frac{2}{3\sqrt{337}}} \mat{3 \\ -3 \\ 6} \\
                    = \mat{2 \\ \frac{77}{\sqrt{337}}} \\
                    \mat{x_1 \\ x_2} = \mat{\frac{229}{\sqrt{2}} \\ 231}
                \end{gathered}
            \end{displaymath}
        \end{enumerate}
    \end{tasks}
\end{document}