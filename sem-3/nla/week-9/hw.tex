\documentclass{article}

\usepackage{../../classes/kiu}

\usepackage{xcolor}
\usepackage[normalem]{ulem}
\definecolor{lightgray}{RGB}{250, 160, 160}
\usepackage{mathtools}

\DeclareMathOperator{\adj}{adj}

\begin{document}
    \kiuheader{Homework}
    {Numerical Linear Algebra}
    {Dimitri Tabatadze}

    \begin{enumerate}
        \item {
            We can rewrite the system in matrix form:
            \begin{displaymath}
                \begin{cases}
                    2x_1 + \phantom{2}x_2 \phantom{\phantom{}+ 3x_3} = 3 \\
                    \phantom{2}x_1 + 2x_2 + \phantom{3}x_3 = -2 \\
                    \phantom{2_x1 + \phantom{}} 2x_2 + 3x_3 = 0
                \end{cases}
                \implies
                \mat{
                    2 & 1 & 0 \\ 1 & 2 & 1 \\ 0 & 2 & 3
                }
                \mat{x_1 \\ x_2 \\ x_3} = \mat{3 \\ -2 \\ 0}.
            \end{displaymath}
            Now we do \(LU\) factorization on the matrix:
            \begin{displaymath}
                \mat{
                    2 & 1 & 0 \\ 1 & 2 & 1 \\ 0 & 2 & 3
                } = LU = \mat{
                    \alpha_1 & 0 & 0 \\ 1 & \alpha_2 & 0 \\ 0 & 2 & \alpha_3
                }\mat{
                    1 & \beta_1 & 0 \\ 0 & 1 & \beta_2 \\ 0 & 0 & 1
                }
            \end{displaymath}
            where
            \begin{displaymath}
                \begin{aligned}
                    1 \cdot \alpha_1 &= 2 
                    &\implies& \alpha_1 = 2 \\
                    \beta_1 \cdot \alpha_1 &= \beta_1 \cdot2  = 1 
                    &\implies& \beta_1 = 1/2 \\
                    \beta_1 + \alpha_2 &= 1/2 + \alpha_2 = 2 
                    &\implies& \alpha_2 = 3/2 \\
                    \beta_2 \cdot \alpha_2 &= \beta_2 \cdot 3/2
                    &\implies& \beta_2 = 2/3 \\
                    2\beta_2 + \alpha_3 &= 4/3 + \alpha_3 = 3
                    &\implies& \alpha_3 = 5/3
                \end{aligned}
            \end{displaymath}
            giving us
            \begin{displaymath}
                L = \mat{2 & 0 & 0 \\ 1 & 3/2 & 0 \\ 0 & 2 & 5/3},
                U = \mat{1 & 1/2 & 0 \\ 0 & 1 & 2/3 \\ 0 & 0 & 1}.
            \end{displaymath}

            Now, we do back-substiturion and forward-substiturion separately on \(U\) and \(L\).

            \begin{displaymath}
                \begin{aligned}
                    U\vec{t} = 
                    \mat{1 & 1/2 & 0 \\ 0 & 1 & 2/3 \\ 0 & 0 & 1}
                    \mat{t_1 \\ t_2 \\ t_3} &=
                    \mat{3 \\ -2 \\ 0} \implies
                    \mat{t_1 \\ t_2 \\ t_3} =
                    \mat{4 \\ -2 \\ 0} & \text{(via back-substiturion)} \\
                    LU\vec{t} = 
                    \mat{2 & 0 & 0 \\ 1 & 3/2 & 0 \\ 0 & 2 & 5/3}
                    \mat{x_1 \\ x_2 \\ x_3} &= 
                    \mat{4 \\ -2 \\ 0} \implies
                    \mat{x_1 \\ x_2 \\ x_3} = 
                    \mat{2 \\ -8/3 \\ 16/5} & \text{(via forward-substiturion)}
                \end{aligned}
            \end{displaymath}
        }
        \item{
            \begin{enumerate}
                \item {
                    \begin{displaymath}
                        \begin{gathered}
                            \mat{3&1&2\\6&3&4\\3&1&5} \overset{r_2-2r_1, r_3-r_1}{\longrightarrow}
                            \mat{3&1&2\\0&1&0\\0&0&3} \\
                            \mat{3&1&2\\6&3&4\\3&1&5} =
                            \mat{1&0&0\\2&1&0\\1&0&1}
                            \mat{3&1&2\\0&1&0\\0&0&3}
                        \end{gathered}
                    \end{displaymath}
                }
                \item {
                    \begin{displaymath}
                        \begin{gathered}
                            \mat{1&-1&0\\2&2&3\\-1&3&2} \overset{r_2-2r_1, r_3+r_1}{\longrightarrow}
                            \mat{1&-1&0\\0&4&3\\0&2&2} \overset{r_3-r_2\cdot1/2}{\longrightarrow}
                            \mat{1&-1&0\\0&4&3\\0&0&1/2}
                            \\
                            \mat{1&-1&0\\2&2&3\\-1&3&2} =
                            \mat{1&0&0\\2&1&0\\-1&1/2&1}
                            \mat{1&-1&0\\0&4&3\\0&0&1/2}
                        \end{gathered}
                    \end{displaymath}
                }
            \end{enumerate}
        }
        \item {
            \begin{enumerate}
                \item {
                    \begin{displaymath}
                        PA =
                        \mat{0&1&0\\1&0&0\\0&0&1}
                        \mat{0&2&-1\\1&-1&2\\1&-1&4} =
                        \mat{1&-1&2\\0&2&-1\\1&-1&4}
                    \end{displaymath}
                }
                \item {
                    \begin{displaymath}
                        PA =
                        \mat{1&0&0&0\\0&0&0&1\\0&0&1&0\\0&1&0&0}
                        \mat{1&1&-1&2\\-1&-1&1&5\\2&2&3&7\\2&3&4&5} =
                        \mat{
                            1&1&-1&2\\
                            2&3&4&5\\
                            2&2&3&7\\
                            -1&-1&1&5\\
                        }
                    \end{displaymath}
                }
            \end{enumerate}
        }
        \item {
            \begin{displaymath}
                \begin{gathered}
                    \mat{4&2&0\\4&4&2\\2&2&3}
                    \overset{r_2-r_1, r_3-r_2\cdot1/2}{\longrightarrow}
                    \mat{4&2&0\\0&2&2\\0&0&2}\\
                    D = \mat{4&0&0\\0&2&0\\0&0&2},
                    U = \mat{1&1/2&0\\0&1&1/2\\0&0&1},
                    L = \mat{1&0&0\\1&1&0\\0&2&1}
                \end{gathered}
            \end{displaymath}
        }
    \end{enumerate}
\end{document}