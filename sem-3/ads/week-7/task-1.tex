For \(h\) to be a ring homomorphism, it needs to satisfy the two conditions:
\begin{gather*}
    h(a + b) = h(a) +' h(b) \\
    h(a * b) = h(a) *' h(b)
\end{gather*}

\paragraph{Addition.}
\begin{displaymath}
    \begin{gathered} 
        a, b \in S_n,
        (a +_n b)(i, j) = a(i, j) + b(i, j) \\
        \begin{aligned}
            h(a +_n b)(p, q)(i, j)
            =\phantom{}& (\CR a +_n \CG b)\CB{(i + (p - 1)(n / 2), j + (q - 1)(n / 2))} \\
            =\phantom{}& \CR a\CB{(i + (p - 1)(n / 2), j + (q - 1)(n / 2))}\ + \\
             & \CG b\CB{(i + (p - 1)(n / 2), j + (q - 1)(n / 2))} \\
            =\phantom{}& h(\CR a)(p, q)(i, j) + h(\CG b)(p, q)(i, j)
        \end{aligned}
    \end{gathered}
\end{displaymath}
where \(\quad p, q \in \{1, 2\}, \quad i, j \in [1:n/2]\).

\paragraph{Multiplication.}
\begin{displaymath}
    \begin{gathered}
        a, b \in S_n,
        (a +_n b)(i, j) = a(i, j) + b(i, j) \\
        \begin{aligned}
            h(a *_n b)(p, q)(i, j)
            &= (a *_n b)(\CR{i + (p - 1)(n / 2)}, \CG{j + (q - 1)(n / 2)}) \\
            &= \sum_{\CM t = 1}^{n} a(\CR{i + (p - 1)(n / 2)}, \CM t) * b(\CM t, \CG{j + (q - 1)(n / 2)}) \\
            &= \sum_{\CB k = 1}^{2} \sum_{\CM t = 1}^{n / 2} a(\CR{i + (p - 1)(n / 2)}, \CM t + (\CB k - 1) \cdot (n / 2)) * b(\CM t + (\CB k - 1) \cdot (n / 2), \CG{j + (q - 1)(n / 2)}) \\
            &= \sum_{\CB k = 1}^{2} \sum_{\CM t = 1}^{n / 2} h(a)(p-1, (\CB k - 1))(i, \CM t) * h(b)((\CB k - 1), q-1)(\CM t, j) \\
            &= \left(\sum_{\CB k = 1}^{2} h(a)(p-1, (\CB k - 1)) *_{n/2} h(b)((\CB k - 1), q-1)\right)(i, j) \\
            &= \left(h(a) *_{n/2, 2} h(b)\right)(p, q)(i, j) \\
        \end{aligned}
    \end{gathered}
\end{displaymath}

\paragraph{Multiplicative identity.}
\begin{displaymath}
    \begin{gathered}
        h(I_n) = \mat{I_{n/2} & 0_{n/2} \\ 0_{n/2} & I_{n/2}} \\
        h(I_n)(p, q)(i, j) = I_n(i + (p - 1)(n / 2), j + (q - 1)(n / 2)) \implies h(I_n) = I_{n/2,2}
    \end{gathered}
\end{displaymath}