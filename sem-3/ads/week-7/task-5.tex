First, for a decimal counter with \(k\) digits we set
\begin{itemize}
    \item \(c_i\): number of digits changed
    \item \(t_i\): number of trailing 9s after \(op_i\)
\end{itemize}
and
\begin{itemize}
    \item \(\Phi_i\): \(\text{sum of the digits}\over 9\)
    \item \(\Phi_i > 0\): all trailing 9s become 0, but the digit just left of them gets incremented
    \begin{displaymath}
        \Phi_i = \Phi_{i-1} - t_{i-1} + \frac{1}{9}
    \end{displaymath}
    \item \(\Phi_i = 0\): every digit could have just turned from 9 to 0.
    \begin{displaymath}
        t_{i-1} = \Phi_{i-1} = k, \quad \Phi_i = \Phi_{i-1} - t_{i-1} < \Phi_{i-1} - t_{i-1} + \frac{1}{9}
    \end{displaymath}
\end{itemize}

From this, we can infer that
\begin{displaymath}
    \begin{aligned}
        \CG{\Phi_i - \Phi_{i-1}}
        &\leq \frac{1}{9} - t_{i-1}
    \end{aligned}
\end{displaymath}

It's clear that the cost of the \(i\)-th operation is
\begin{displaymath}
    c_i = \begin{cases}
        t_{i - 1} + 1 & \text{if } t_{i - 1} < k \\
        t_{i - 1} & \text{if } t_{i - 1} = k
    \end{cases}
\end{displaymath}
giving us the upper bound
\begin{displaymath}
    \CR{c_i} \leq t_{i - 1} + 1.
\end{displaymath}

Now we just do some arithmetic
\begin{displaymath}
    \begin{aligned}
        \hat{c}_i
        &= \CR{c_{i}} + \CG{\Phi_i - \Phi_{i-1}} \\
        &\leq t_{i - 1} + 1 + \frac{1}{9} - t_{i-1} \\
        &= \CM{\frac{10}{9}}
    \end{aligned}
\end{displaymath}

% \pgfornament[anchor=south,width=1cm]{78}

\begin{displaymath}
    \boxed{
        \sum_{i = 1}^n c_i 
        \leq \sum_{i = 1}^n \CM{\frac{10}{9}} + \CB{\underbrace{\Phi_0 - \Phi_n}_{\text{at most }k}}
        \leq \CM{\frac{10}{9}}n + \CB k
    }
\end{displaymath}