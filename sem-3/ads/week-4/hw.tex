\documentclass{article}

\usepackage{../../classes/dim}
\usepackage{listings}
\usepackage{graphicx}               % Necessary to use \scalebox
\usepackage[dvipsnames]{xcolor}               % Necessary to use \scalebox

\begin{document}
    \header{Homework}
    {Algorithms and Data Structures}

    \begin{center}
        \large
        Worksheet 4
    \end{center}
    \begin{tasks}
        \item {
            \begin{proof} By induction on \(i\).
                Base case: \(i = 1\)
                \begin{displaymath}
                    p_b(b(j), c) = 1 = 1!
                \end{displaymath}
                Induction step:
                \begin{displaymath}
                    p_b(b(j), c) 
                    = \frac{1}{i} \cdot p_{b-b(j)}(c)
                    = \frac{1}{i} \cdot \frac{1}{(i-1)!} 
                    = \frac{1}{i!}
                \end{displaymath}
            \end{proof}
        }
        \item {
            They probably used the fact that \((a+b)^2 = a^2 + b^2 + 2ab\). They could calculate the sum of \(a\) and \(b\) and the squares \(a^2, b^2, (a+b)^2\) fairly easily, which in (also fairly easy) combination \((a+b)^2 - a^2 - b^2\) would give them the double of the product they were looking for \(2ab\) --- which they would also be able to halve easily to obtain \(ab\).
        }
        \item {
            \begin{verbatim}
let p = random(0, size(A));
let L = [];
let R = [];
for i in 0..size(A) {
    if i != p {
        if A[i] < A[p] { L.push(A[i]) }
        else { R.push(A[i]) }
    }
}
            \end{verbatim}
        }
        \item {
            First of all, this definition is not full. The case where either \(n = 0\) or \(m = 0\) is not handeled. I can not make an assumption fot what sould be done, since it is very important at least for the 2-nd task, so I will just assume that \(n = m\)
            \begin{enumerate}[label={(\arabic*)}]
                \item {
                    We can write the number of comparisons as \(t'(n, m) = 1 + t'(n-1, m-1)\) which can be shortened to \(t'(n, m) = n = m\). Whereas, for the original merge \(t(n, m) = n + m - 1\). This modified merge has less comparisons.
                }
                \item {
                    This can be disproven with a counterexample: Let \(A = [1, 2, 3], B = [4, 5, 6]\), then \(\text{merge}'(A, B) = [1, 4, 2, 5, 3, 6]\) which is not sorted.
                }
            \end{enumerate}
        }
        \item {
            \begin{enumerate}[label={(\arabic*)}]
                \item {
                    We can say that \(S_n = [1:365]^n\). Thereby \(p_n(x) = 365^{-n}\)
                }
                \item {
                    \begin{displaymath}
                        \#E_n = 365 \cdot 364 \cdot \ldots \cdot (365-n +1) = \frac{365!}{(365-n)!}
                    \end{displaymath}
                    This also introduces a constraint \(n \leq 365\) which is quite logical -- you can't have more people with distinct birthdays than the number of possible birthdays.
                }
                \item {
                    \begin{displaymath}
                        p(E_n) = \frac{\#E_n}{365^n} = \frac{365!}{365^n(365-n)!}
                    \end{displaymath}
                }
                \item {
                    \begin{displaymath}
                        p(E_{23}) 
                        = \frac{365!}{(365-23)!}\frac{1}{365^{23}}
                        = \frac{342\cdot343\cdot\ldots\cdot364}{365^{22}}
                        \approx \frac{3699}{7509} \approx 0.4927...
                    \end{displaymath}
                }
                \item {{\color{Sepia}{\(p(E_n)\)is}} the probability of there being no pair in \(n\) people that share a birthday. What was probably ment to be written in the worksheet was the statement \(1-p(E_{23})>1/2\), which means that given 23 people, the chance of there being a pair sharing a birthday is more than 50\%.}
            \end{enumerate}
        }
    \end{tasks}
\end{document}