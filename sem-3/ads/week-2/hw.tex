\documentclass{article}

\usepackage[margin=4cm]{geometry}

\usepackage{amsmath}
\usepackage{amsfonts}
\usepackage{amssymb}
\usepackage{enumitem}
\usepackage{mathtools}

\usepackage{listings}
\usepackage{xcolor}
\definecolor{codegreen}{rgb}{0,0.6,0}
\definecolor{codegray}{rgb}{0.5,0.5,0.5}
\definecolor{keyword}{rgb}{0.1,0.1,0.7}
\definecolor{codepurple}{rgb}{0.58,0,0.82}
\definecolor{red}{rgb}{0,0.7,0.3}
\lstdefinestyle{mystyle}{
    commentstyle=\color{codegreen},
    keywordstyle=\color{keyword},
    numberstyle=\tiny\color{codegray},
    stringstyle=\color{codepurple},
    basicstyle=\ttfamily\footnotesize,
    captionpos=b,
    keepspaces=true,
    numbers=left,
    showspaces=false,
    showstringspaces=false,
    frame=single,
    numbersep=1.5em,
    tabsize=6,
	belowskip=0em,
}
\lstset{style=mystyle}

\title{A\&DS Worksheet 2 solutions}
\author{Dimitri Tabatadze}

\begin{document}
    \maketitle

    \begin{enumerate}
        \item {
            By induction on n, it can be proven that
            \begin{displaymath}
                \left\lceil\frac{n}{2}\right\rceil = \left\lfloor\frac{n+1}{2}\right\rfloor.
            \end{displaymath}

            \paragraph{Base case.} \(n=1, n=2\)
            \begin{displaymath}
                \begin{aligned}
                    \left\lceil\frac{1}{2}\right\rceil &= \left\lfloor\frac{1+1}{2}\right\rfloor \implies 1 = 1\\
                    \left\lceil\frac{2}{2}\right\rceil &= \left\lfloor\frac{2+1}{2}\right\rfloor \implies 1 = 1
                \end{aligned}
            \end{displaymath}

            \paragraph{Induction step.} \(n\to n+2\)
            \begin{displaymath}
                \left\lceil\frac{n+2}{2}\right\rceil 
                = \left\lceil\frac{n}{2} + 1\right\rceil 
                = \left\lceil\frac{n}{2}\right\rceil + 1 
                = \left\lfloor\frac{n+1}{2}\right\rfloor + 1 
                = \left\lfloor\frac{n+1}{2} + 1\right\rfloor
                = \left\lfloor\frac{n+2+1}{2}\right\rfloor.
            \end{displaymath}

            Same goes for
            \begin{align*}
                \frac{n}{2} - \frac{1}{2} &\leq \left\lfloor\frac{n}{2}\right\rfloor \leq \frac{n}{2} \Longleftrightarrow \\
                \frac{n - 1}{2} &\leq \left\lfloor\frac{n}{2}\right\rfloor \leq \frac{n}{2} \\
            \end{align*}

            \paragraph{Base case.} \(n=1, n=2\)
            \begin{displaymath}
                \begin{aligned}
                    \frac{1 - 1}{2} &\leq \left\lfloor\frac{1}{2}\right\rfloor \leq \frac{1}{2} \implies 0 \leq 0 \leq \frac{1}{2} \\
                    \frac{2 - 1}{2} &\leq \left\lfloor\frac{2}{2}\right\rfloor \leq \frac{2}{2} \implies \frac{1}{2} \leq 1 \leq 1 \\
                \end{aligned}
            \end{displaymath}

            \paragraph{Induction step.} \(n\to n+2\)
            \begin{displaymath}
                \begin{aligned}
                    \frac{(n + 2) - 1}{2} &\leq \left\lfloor\frac{(n + 2)}{2}\right\rfloor \leq \frac{(n + 2)}{2} & \Longleftrightarrow \\
                    \frac{n - 1}{2} + 1 &\leq \left\lfloor\frac{n}{2} + 1\right\rfloor \leq \frac{n}{2} + 1 & \Longleftrightarrow \\
                    \frac{n - 1}{2} + 1 &\leq \left\lfloor\frac{n}{2}\right\rfloor + 1 \leq \frac{n}{2} + 1 & \Longleftrightarrow \\
                    \frac{n - 1}{2} &\leq \left\lfloor\frac{n}{2}\right\rfloor \leq \frac{n}{2}
                \end{aligned}
            \end{displaymath}

            We used this to prove lemma 1 on slide 9. Lemma 1 says that every step of 
        }
        \item {
            It can be proven that a binary tree with depth \(d\) has at most \(2^d\) leaves by induction on \(d\).
            
            Let \(n = 2^d\) be the induction hypothesis.

            \paragraph{Base case. \(d=0\)}
            A binary tree with depth \(d = 0\) can only consist of a single node, which is both the root and the only leaf of the tree. Hence, this tree has one leaf \(n = 2^0\).

            \paragraph{Induction step.}
            If we had a tree with \(d\) depth and \(n = 2^d\) leaves, we can give every leaf two (the maximum possible amount) children. This will increase the depth to \(d+1\) and double the leaf count giving us \(n = 2^{d+1}\).
        }
        \item { 
            For the given data, we can say that
            \begin{displaymath}
                \begin{aligned}
                    W &= (S,p) \\
                    S &= \left\{{(\text{M},\text{S}), (\text{M},\text{DNS}), (\text{F},\text{S}), (\text{F},\text{DNS})}\right\} \\
                    p(x) &= \begin{cases}
                        0.18 & \text{if } x = (\text{M},\text{S}) \\
                        0.31 & \text{if } x = (\text{M},\text{DNS}) \\
                        0.14 & \text{if } x = (\text{F},\text{S}) \\
                        0.37 & \text{if } x = (\text{F},\text{DNS})
                    \end{cases}
                \end{aligned}
            \end{displaymath}
            \begin{enumerate}[label=(\arabic*)]
                \item \(\frac{0.18}{0.32} = \frac{9}{16}\)
                \item \(\frac{0.18}{0.49} = \frac{18}{49}\)
                \item \(\frac{0.37}{0.68} = \frac{37}{68}\)
            \end{enumerate}
        }
        \newpage
        \item {
            \begin{enumerate}[label=(\arabic*)]
                \item {
                    Let \(W_1 = (S_1, p_1)\) be the probability space of a 6 sided die, then
                    \begin{displaymath}
                        p_1(i) = \frac{1}{6}, \ i \in S_1
                    \end{displaymath}
                    where \(S_1 = \{1, 2, 3, 4, 5, 6\}\). The probability space of two independent
                    six sided dice would then be \(W = W_1\times W_1 = (S, p)\). Now, with 
                    \(A_i = \{(a, b) | a+b = i\}\) the probability \(p(A_i)\) can easily be defined as
                    \begin{displaymath}
                        p(A_i) = |A_i| \cdot \frac{1}{36}
                    \end{displaymath}
                    where \(|A_i|\) is the size of the set \(A_i\), which can also be written as
                    \begin{displaymath}
                        |A_i| = 6 - |i - 7|
                    \end{displaymath}
                    so we get
                    \begin{displaymath}
                        p(A_i) = \frac{6 - |i - 7|}{36}
                    \end{displaymath}
                }
                \item {
                    \(E(a + b) = E(a) + E(b)\)
                }
            \end{enumerate}
        }
        \newpage
        \item {
            Custom linked list class:

            \hspace*{2ex} \begin{minipage}[t]{0.8\textwidth}
                \lstinputlisting[language=Java]{src/Mergesort/LinkedList.java}
            \end{minipage}
            \begin{enumerate}[label=(\arabic*)]
                \item {
                    \hspace*{2ex} \begin{minipage}[t]{0.8\textwidth}
                        \lstinputlisting[language=Java]{src/Mergesort/One.java}
                    \end{minipage}
                }
                \item {
                    \hspace*{2ex} \begin{minipage}[t]{0.8\textwidth}
                        \lstinputlisting[language=Java]{src/Mergesort/Two.java}
                    \end{minipage}
                }
                \item {
                    \hspace*{2ex} \begin{minipage}[t]{0.8\textwidth}
                        \lstinputlisting[language=Java]{src/Mergesort/Three.java}
                    \end{minipage}
                }
            \end{enumerate}
        }
    \end{enumerate}
\end{document}